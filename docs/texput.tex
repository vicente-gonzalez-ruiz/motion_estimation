% Emacs, this is -*-latex-*-

\title{Optical Flow Motion Estimation}

\maketitle

\section{ME in a transformed domain}

\begin{figure}
  %\begin{tabular}{cccccc}
  %  \png{one} & \png{x} & \png{y} & \png{x2} & \png{y2} & \png{xy} \\
  %\end{tabular}
  \begin{tabular}{cccccc}
    \png{one}{200} & \png{x}{200} & \png{y}{200} & \png{x2}{200} & \png{y2}{200} & \png{xy}{200} \\
    No motion & Constant velocity in $X$ & Constant velocity in $Y$ & Constant acceleration in $X$ & Constant acceleration in $Y$ & Constant accelarion in diagonal
  \end{tabular}
  \caption{Correlation kernels (basis functions) used by the
    \emph{polynomial expansion} of the Farneb{\"a}ck's ME
    algorithm. See \href{https://github.com/Sistemas-Multimedia/Sistemas-Multimedia.github.io/blob/master/milestones/09-ME/farneback_ME.ipynb}{this}. The analized motion is depicted below the plot of each basis.}
  \label{fig:FarnebacK_basis}
\end{figure}

The motion can be estimated also in a transformed domain. One of these
estimators is the Farneb{\"a}ck's algorithm~\cite{farneback2003two},
which uses the transform defined by the basis functions
\begin{equation}
    \{1, x, y, x^2, y^2, xy\}
\end{equation}
(see the Figure~\ref{fig:FarnebacK_basis}). In this transform domain,
which is applied by overlapped regions, the corresponding subbands
quantify the tendency of the image to increase its intensity in
different 2D directions, and therefore, it is more efficient to know
the direction in which the objects are moving.

\begin{figure}
  \centering
  \png{stockholm_hat_P_farneback}{800}
  \caption{The prediction frame (${\hat{\mathbf P}}$). See \href{https://github.com/Sistemas-Multimedia/Sistemas-Multimedia.github.io/blob/master/milestones/09-ME/farneback_ME.ipynb}{this}.}
  \label{fig:hat_P_farneback}
\end{figure}

\begin{figure}
  \centering
  \png{stockholm_error_farneback}{800}
  \caption{The prediction error frame (${\mathbf R} - {\hat{\mathbf P}}$). See \href{https://github.com/Sistemas-Multimedia/Sistemas-Multimedia.github.io/blob/master/milestones/09-ME/farneback_ME.ipynb}{this}.}
  \label{fig:error_farneback}
\end{figure}

\begin{figure}
  \centering
  \png{stockholm_MVs_farneback}{800}
  \caption{Motion vectors to map ${\mathbf P}$ (from which each pixel has been mapped) onto ${\mathbf R}$. See \href{https://github.com/Sistemas-Multimedia/Sistemas-Multimedia.github.io/blob/master/milestones/09-ME/farneback_ME.ipynb}{this}.}
  \label{fig:MVs_farneback}
\end{figure}

The Farneback's ME is a dense ME, and it provides subpixel
interpolation because the MVs are real numbers (see the
Figures~\ref{fig:hat_P_farneback}, \ref{fig:error_farneback} and
\ref{fig:MVs_farneback}). Notice that the prediction is the best of
the all tested algorithms, probably by the subpixel accuracy.
